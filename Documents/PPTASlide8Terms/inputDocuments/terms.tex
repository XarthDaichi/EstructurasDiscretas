\section{Usabilidad}

\begin{frame}{Usabilidad (normalmente tiene prioridad)}

    \begin{block}{Definición del profe}
        "facilidad de uso" (maximizar la satisfacción del usuario con un producto)
    \end{block}
    \begin{alertblock}{\cite{ISO25010}}
        "Degree to which a product or system can be used by specified users to achieve specified goals with effectiveness, efficiency and satisfaction in a specified context to use."
    \end{alertblock}
    \begin{exampleblock}{\cite{DromeyR.G1995Amfs}}
        "is concerned with the quality of the user interface, its design and performance characteristics."
    \end{exampleblock}
    
\end{frame}

\section{Correctitud}

\begin{frame}{Correctitud}

    \begin{block}{Definición del profe}
        "La funcionalidad es la esperada"
    \end{block}
    \begin{alertblock}{\cite{ISO25010}}
        "Degree to which the set of functions covers all the specified tasks and user objectives."
    \end{alertblock}
    \begin{exampleblock}{\cite{DromeyR.G1995Amfs}}
        Afecta funcionalidad y fiabilidad. Sus propiedades son: Computable, completo, asignado (variables tienen valor antes de uso), preciso, inicializado (tareas para las variables \textit{"loop"}), progresivo, variante y consistente.
    \end{exampleblock}
\end{frame}


\section{Rendimiento}

\begin{frame}{Rendimiento}
    \begin{block}{Definición del profe}
        "\space uso eficiente de recursos"
    \end{block}
    \begin{alertblock}{\cite{ISO25010}}
        "This characteristic representa the performance relative to the amount of resources used under stated conditions."
    \end{alertblock}
\end{frame}

\begin{frame}{Rendimiento: tiempo}
    \begin{block}{\cite{ISO25010}}
        "Degree to which the response and precessing times and throughput rates of a product or system, when performing its functions, meet requirements."
    \end{block}
\end{frame}

\begin{frame}{Rendimiento: memoria}
    \begin{block}{\cite{ISO25010}}
        "Degree to which the maximum limits of a product or system parameter meet requirements"
    \end{block}
    Nota: esto estaba bejo el término de \textit{capacity} asumiendo capacidad de memoria cumple que hay limitante de producto y requerimientos.
\end{frame}

\section{Mantenibilidad}

\begin{frame}{Mantenibilidad}
    \begin{block}{Definición del profe}
        "\space capacidad de evolucionar ante cambios"
    \end{block}
    \begin{alertblock}{\cite{ISO25010}}
        "This characteristic represents the degree of effectiveness and efficiency with which a product or system can be modified to improve it, correct it or adapt it to changes in environment, and in requirements."
    \end{alertblock}
\end{frame}

\section{Reuso/Extensibilidad}

\begin{frame}{Reuso/Extensibilidad}
    \begin{block}{Definición del profe}
        "\space capacidad de reutilizar partes sin cambiarlas"
    \end{block}
    \begin{alertblock}{\cite{ISO25010}}
        "Degree to which an asset can be used in more than one system, or in building other assets."
    \end{alertblock}
\end{frame}

\section{Modularidad}

\begin{frame}{Modularidad}
    \begin{block}{Definición del profe}
        "descomposición adecuada en partes coherentes y desacoplados"
    \end{block}
    \begin{alertblock}{\cite{ISO25010}}
        "Degree to which a system or computer program is composed of discrete components such that a change to one component has minimal impact on other components."
    \end{alertblock}
\end{frame}

\section{Robustez}

\begin{frame}{Robustez}
    \begin{block}{Definición del profe}
        "\space reacción/manejo apropiado ante/de errores"
    \end{block}
    \begin{alertblock}{\cite{ISO25010}}
        Fault tolerance: "Degree to which a system, product or component operates as intended despite the presence of hardware or software faults."
    \end{alertblock}
    \begin{exampleblock}{\cite{2002Fose}}
        \begin{quoting} se comporta en forma razonable aún en circunstancias que no fueron anticipadas en la especificación de requerimientos \end{quoting}
    \end{exampleblock}
\end{frame}

\section{Enfoques}

\begin{frame}{Enfoques: Macro: Arquitectura}
\begin{block}{\cite{Asale2022Mar}}
    \begin{quoting} Estructura lógica y física de los componentes de una computadora. \end{quoting}
\end{block}
\end{frame}

\begin{frame}{Enfoques: Micro: algoritmos y estructuras de datos}
    \begin{block}{\cite{skiena_2012}}
        Algorithm: a procedure to accomplish a specific task. [...] idea behind any reasonable computer program. [...] must solve a general, well-specified problem."
    \end{block}
\end{frame}