\section{Definiciones y explicaciones}
\frame{\sectionpage}

\begin{frame}{Minimum Spanning Tree o árbol de exploración mínimo}
    \begin{block}{Spanning Tree} % azul
        Según \cite{Goodaire_Parmenter_2001}, es un subgrafo de un grafo conectado que incluye todos los vertices
    \end{block}
    \begin{exampleblock}{Minimum Spanning Tree} %verde
        Es un subgrafo de un grafo con peso, pero con el mínimo peso posible
    \end{exampleblock}

    % alertblock es rojo, theorem es como block pero en indenta y demostration es igual que block
\end{frame}

\begin{frame}{Heurístico y Greedy}
    \begin{columns}
        \begin{column}{0.5\textwidth}
            \textcolor{yellow}{Algoritmo Heurístico:} Como visto en clase el algoritmo heurístico es el algoritmo que va haciendo decisiones sin conocimiento de lo siguiente. O como dicen en este foro (\cite{heuri}), es "Las heurísticas implican utilizar un enfoque de aprendizaje y descubrimiento para alcanzar una solución"
        \end{column}
        \begin{column}{0.5\textwidth}
            \textcolor{yellow}{Algoritmo Greedy:} Como se vio en clase el algoritmo greedy es un algoritmo heurístico que no hace backtracking.
            \begin{block}{Algoritmo de Kruskal}
                El algoritmo de Kruskal es un algoritmo heurístico y greedy para buscar el árbol de exploración de peso mínimo
            \end{block}
        \end{column}
    \end{columns}
\end{frame}
